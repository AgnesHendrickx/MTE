\subsection{Visualisation}
As our final objective is to compare the observed to the measured values, it is desirable to display the computed values in plots that include the field data points and the trajectory of the path. Therefore, the magnetic field above the flow and the height of the elevation will be plotted against the distance between the measurement points. Such plots are easily comparable on all slopes and at all sites. This does affect the reliability of the displayed topography in these plots; only if the path is perpendicular to the slope would this display the topography perfectly. Undoubtedly, this was not always possible or achieved in the field. However, as one of the selection criteria for the paths was for it to be perpendicular to the slope, we trust these plots to hold an accurate representation of the salient topographic features.  \par
Discrepancies in height data between the DEM and field data are not uncommon. Multiple factors can contribute to such misfit, including inaccuracies in the field data or DEM, multiple coordinate conversions, potential discrepancies between coordinate systems, and more. To quantify this disparity, we computed the height difference across all field path points for both DEMs, averaged them, and then adjusted the field measurement paths accordingly (refer to the code for exact values). Ideally, one constant offset would be applicable across all sites, but this wasn't the reality. Nonetheless, our primary concern is aligning heights for accurate comparison. We aren't particularly invested in identifying the genuine offset or its root cause. Hence, applied offsets might vary by site and were deducted from all field measurement points prior to plotting. Any height offset between values of the 2 and 5m DEM were also compensated in similar fashion. \par
Additionally, in some instances, a clear spatial misalignment was evident, with the field topography and DEM aligning better after a minor adjustment. Any adjustments made for visualization clarity were applied manually, grounded in enhancing the congruence of topographic features. Whenever modifications were made, the figures are labeled as "shifted", and the original unaltered plots can be found in the appended section.
